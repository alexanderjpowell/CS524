%%%%%%%%%%%%%%%%%%%%%%%%%%%%%%%%%%%%
% Alexander Powell
% CSCI 524 - Computer Architecture
% Prof: Adwait Jog
% Homework #6
% Due: 11.10.2016
%%%%%%%%%%%%%%%%%%%%%%%%%%%%%%%%%%%%

\documentclass[10pt]{article} %
\usepackage{fullpage}
\usepackage{graphicx}
\usepackage{graphics}
\usepackage{psfrag}
\usepackage{amsmath,amssymb}
\usepackage{enumerate}

\setlength{\textwidth}{6.5in}
\setlength{\textheight}{9in}

\newcommand{\cP}{\mathcal{P}}
\newcommand{\N}{\mathbb{N}}
\newcommand{\Z}{\mathbb{Z}}
\newcommand{\R}{\mathbb{R}}
\newcommand{\Q}{\mathbb{Q}}
\newcommand{\points}[1]{{\it (#1 Points)}}
\newcommand{\tpoints}[1]{{\bf #1 Total points.}}

\title{CSCI 524 -- Computer Architecture \\
Homework 6 \\
{\large{\bf Due: November 10, 2016}}}
\date{}
\author{Alexander Powell}


\begin{document}
\maketitle
\begin{enumerate}

\item %1

\begin{enumerate}[a)]
\item %a

The table below shows whether each address in the reference string is a cache hit or miss.  

\begin{center}
\begin{tabular}{| l | l | l | c |}
\hline 
Value & Binary & Hit/Miss & Block \\
\hline 
$1$ & 0001 & Hit & 0001 \\
$18$ & 10010 & Miss & 0010 \\
$2$ & 0010 & Miss & 0010 \\
$3$ & 0011 & Hit & 0011 \\
$4$ & 0100 & Hit & 0100 \\
$20$ & 10100 & Miss & 0100 \\
$5$ & 0101 & Hit & 0101 \\
$21$ & 10101 & Miss & 0101 \\
$33$ & 100001 & Miss & 0001 \\
$34$ & 100010 & Miss & 0010 \\
$1$ & 0001 & Miss & 0001 \\
$4$ & 0100 & Miss & 0100 \\
\hline 
\end{tabular}
\end{center}

So, there were $4$ cache hits and $8$ misses so the cache hit rate is $\frac{1}{3}$.  
This table shows the contents of the cache after the last reference.  

\begin{center}
\begin{tabular}{| c | c | c | c | c | c | c | c | c |}
\hline
Index & 0000 & 0001 & 0010 & 0011 & 0100 & 0101 & 0110 & 0111 \\
\hline
Value & 0    & 1    & 34   & 3    & 4    & 21   & 6    & 7    \\
\hline
Index & 1000 & 1001 & 1010 & 1011 & 1100 & 1101 & 1110 & 1111 \\
\hline
Value & 8    & 9    & 10   & 11   & 12   & 13   & 14   & 15   \\
\hline
\end{tabular}
\end{center}

\item %b

The table below shows whether each address in the reference string is a cache hit or miss.  

\begin{center}
\begin{tabular}{| l | l | l | c |}
\hline 
Value & Binary & Hit/Miss & Block \\
\hline 
$1$ & 0001 & Miss & 000 \\
$18$ & 10010 & Miss & 001 \\
$2$ & 0010 & Miss & 001 \\
$3$ & 0011 & Hit & 001 \\
$4$ & 0100 & Miss & 010 \\
$20$ & 10100 & Miss & 010 \\
$5$ & 0101 & Miss & 010 \\
$21$ & 10101 & Miss & 010 \\
$33$ & 100001 & Miss & 000 \\
$34$ & 100010 & Miss & 001 \\
$1$ & 0001 & Miss & 000 \\
$4$ & 0100 & Miss & 010 \\
\hline 
\end{tabular}
\end{center}

There was only $1$ cache hit and $11$ misses, so the hit rate is $\frac{1}{12}$.  
After the last reference the cache looks like the following (a hyphen denotes an empty cache block):

\begin{center}
\begin{tabular}{| c | c | c | c | c | c | c | c | c |}
\hline
Index & 000 & 001 & 010 & 011 \\
\hline
Value & $0$, $1$ & $34$, $35$ & $20$, $21$ & - \\
\hline
Index & 100 & 101 & 110 & 111 \\
\hline
Value & - & - & - & - \\
\hline
\end{tabular}
\end{center}

\item %c

The table below shows whether each address in the reference string is a cache hit or miss.  

\begin{center}
\begin{tabular}{| l | l | l | c |}
\hline 
Value & Binary & Hit/Miss & Block \\
\hline 
$1$ & 0001 & Miss & 000 \\
$18$ & 10010 & Miss & 001 \\
$2$ & 0010 & Miss & 001 \\
$3$ & 0011 & Hit & 001 \\
$4$ & 0100 & Miss & 010 \\
$20$ & 10100 & Miss & 010 \\
$5$ & 0101 & Miss & 010 \\
$21$ & 10101 & Miss & 010 \\
$33$ & 100001 & Miss & 000 \\
$34$ & 100010 & Miss & 001 \\
$1$ & 0001 & Miss & 000 \\
$4$ & 0100 & Miss & 010 \\
\hline 
\end{tabular}
\end{center}

There were $5$ cache hits and $7$ misses, so the hit rate is $\frac{5}{12}$.  
After the last reference the cache looks like the following (a hyphen denotes an empty cache block):

\begin{center}
\begin{tabular}{| c | c | c | c | c | c | c | c | c |}
\hline
Index & 000 000 & 001 001 & 010 010 & 011 011 \\
\hline
Value & $0$, $1$ \text{  } $32$, $33$ & $34$, $35$ \text{  } $2$, $3$ & $4$, $5$ \text{  } $20$, $21$ & - - \\
\hline
Index & 100 100 & 101 101 & 110 110 & 111 111 \\
\hline
Value & - & - & - & - \\
\hline
\end{tabular}
\end{center}

\end{enumerate}

\item %2

\begin{enumerate}[a)]
\item %a

Clock rate is equal to $\dfrac{1}{\text{Hit time}}$, so in this case we have $\dfrac{1}{0.5 \text{ nsec}} = 2 \text{ GHz}$.  

\item %b

The average memory access time is equal to the hit time plus the miss rate multiplied by the miss penalty.  By pluggin in the given values, we get $0.5 + 0.08 \times \frac{100}{.5} = 16.5 \text{ nsec}$.  

The actual CPI can be calculated as $1 + \frac{1}{0.5} \times (0.08 \times (100 - 0.5) \times 0.35) = 6.57$.  

\item %c

Again, the average memory access time is equal to the hit time plus the miss rate multiplied by the miss penalty but we need to take into account both the L1 and L2 caches.  The AMAT is calculated as $0.5 + 0.08 \times (5 + 0.95 * (100 - 5)) = 8.12$.  

Again, to calculate the actual CPI we have $1 + 0.35 \times 0.08 \times \frac{100}{5} = 7.65$.  1+ 1/.5 * .35*.08*95.25

\end{enumerate}

\item %3

The tables below show the performance results of the six benchmarks on the five different machines.  

\textbf{bzip2}
\begin{center}
\begin{tabular}{| c | c | c | c | c | c |}
\hline
 & M1 & M2 & M3 & M4 & M5 \\
\hline
sim IPC & 0.2852 & 0.2674 & 0.4306 & 0.6521 & 0.6521 \\
il1 hits/misses & 2000362/571 & 2000369/571 & 2000407/480 & 2000404/480 & 2000404/480 \\
dl1 hits/misses & 414598/296557 & 414598/296557 & 414600/296557 & 488333/222829 & 488333/222829 \\
ul2 hits/misses & 339576/201147 & 339588/201135 & 33921/201155 & 226995/201251 & 226995/201251 \\
itlb hits/misses & 2000916/17 & 2000923/17 & 2000870/17 & 2000867/17 & 2000867/17 \\
dtlb hits/misses & 666080/45077 & 660080/45077 & 666082/45077 & 666087/45077 & 666087/45077 \\
\hline
\end{tabular}
\end{center}

\textbf{equake}
\begin{center}
\begin{tabular}{| c | c | c | c | c | c |}
\hline
 & M1 & M2 & M3 & M4 & M5 \\
\hline
sim IPC & 0.6489 & 0.4785 & 0.7894 & 0.7439 & 0.7439 \\
il1 hits/misses & 2093198/180319 & 2093881/180319 & 2102027/21086 & 2102861/21086 & 2102861/21086 \\
dl1 hits/misses & 747261/16447 & 747262/16447 & 755807/7441 & 755041/6138 & 755027/6152 \\
ul2 hits/misses & 204515/3359 & 204515/3359 & 30948/3359 & 25926/3359 & 25950/3359 \\
itlb hits/misses & 2273505/12 & 2274188/12 & 2123101/12 & 2123935/12 & 2123935/12 \\
dtlb hits/misses & 772065/58 & 772066/58 & 773896/58 & 773894/58 & 773894/58 \\
\hline
\end{tabular}
\end{center}

\textbf{hmmer}
\begin{center}
\begin{tabular}{| c | c | c | c | c | c |}
\hline
 & M1 & M2 & M3 & M4 & M5 \\
\hline
sim IPC & 0.4751 & 0.4584 & 0.4558 & 0.4353 & 0.4353 \\
il1 hits/misses & 2011833/9127 & 2012044/9127 & 2013122/4545 & 2012352/4545 & 2012423/4545 \\
dl1 hits/misses & 925733/32138 & 925733/32138 & 929051/29605 & 929751/28946 & 929741/28964 \\
ul2 hits/misses & 28189/14633 & 28189/14633 & 20121/14633 & 19240/14633 & 19270/14633 \\
itlb hits/misses & 2020928/32 & 2021139/32 & 2017635/32 & 2016865/32 & 2016936/32 \\
dtlb hits/misses & 957644/227 & 957644/227 & 958430/227 & 958471/227 & 958479/227 \\
\hline
\end{tabular}
\end{center}

\textbf{mcf}
\begin{center}
\begin{tabular}{| c | c | c | c | c | c |}
\hline
 & M1 & M2 & M3 & M4 & M5 \\
\hline
sim IPC & 0.8603 & 0.7122 & 0.9757 & 0.9519 & 0.9518 \\
il1 hits/misses & 2047813/82138 & 2047819/82138 & 2054079/10974 & 2053117/10974 & 2053117/10974 \\
dl1 hits/misses & 987690/118862 & 987690/118862 & 991832/116141 & 992026/115947 & 991997/115976 \\
ul2 hits/misses & 259383/58218 & 259386/58215 & 183425/58214 & 180166/58214 & 180211/58214 \\
itlb hits/misses & 2129939/12 & 2129945/12 & 2065041/12 & 2064079/12 & 2064079/12 \\
dtlb hits/misses & 1105631/921 & 1105632/921 & 1108013/921 & 1108493/921 & 1108493/921 \\
\hline
\end{tabular}
\end{center}

\textbf{milc}
\begin{center}
\begin{tabular}{| c | c | c | c | c | c |}
\hline
 & M1 & M2 & M3 & M4 & M5 \\
\hline
sim IPC & 0.8687 & 0.8848 & 0.8826 & 0.8785 & 0.8784 \\
il1 hits/misses & 2011304/4450 & 2011307/4450 & 2011505/4441 & 2011505/4441 & 2011505/4441 \\
dl1 hits/misses & 992523/137636 & 992523/137637 & 992818/137340 & 992927/137234 & 992908/137253 \\
ul2 hits/misses & 209064/69762 & 209865/68961 & 208677/58214 & 205457/68961 & 205485/68961 \\
itlb hits/misses & 20155740/14 & 2015743/14 & 2015932/14 & 2015932/14 & 2015932/14 \\
dtlb hits/misses & 1128141/2018 & 1128141/2018 & 1128140/2018 & 1128143/2018 & 1128143/2018 \\
\hline
\end{tabular}
\end{center}

\textbf{sjeng}
\begin{center}
\begin{tabular}{| c | c | c | c | c | c |}
\hline
 & M1 & M2 & M3 & M4 & M5 \\
\hline
sim IPC & 0.9311 & 1.0131 & 1.0137 & 0.9415 & 0.9415 \\
il1 hits/misses & 2304395/241 & 2304397/241 & 2264719/221 & 2254268/221 & 225468/221 \\
dl1 hits/misses & 732115/37162 & 732115/37162 & 739869/29542 & 741759/27652 & 741759/27652 \\
ul2 hits/misses & 50599/12410 & 52149/10860 & 40217/10860 & 34377/10860 & 34377/10860 \\
itlb hits/misses & 2304625/11 & 2304627/11 & 2264929/11 & 2254478/11 & 2254478/11 \\
dtlb hits/misses & 769096/181 & 769096/181 & 769230/181 & 769230/181 & 769230/181 \\
\hline
\end{tabular}
\end{center}

When examining the integer benchmarks, like bzip2 and mcf, generally the number of instructions per cycle increased as we made changes to the caches.  However, in some of the experiments, these changes didn't necessarily help.  For example, in the equake benchmark, quadrupling the size of the unified L2 cache caused a performance decrease.  This is due to the higher latencies associated with a larger cache.  It should be noted, however, that performance was gained back in machine 3.  Basically, we should find an optimal point where improvements are made from using a cache with larger storage ability, but whose latencies do not cause performance drops.  Also, generally no changes were observed between machine 4 and 5, where we changed the replacement policy from LRU to FIFO.  In conclusion, we see that by increasing the cache size, the hit rates generally increase as well, although this sometimes comes with the added effect of a higher latency.  

\end{enumerate}

\end{document}















